% Set the Page Layout
\documentclass[12pt]{article}
\usepackage[inner = 2.0cm, outer = 2.0cm, top = 2.0cm, bottom = 2.0cm]{geometry}

% Package to write pseudo-codes
\usepackage{algorithm}

% Don't Remove the 'end' at the end of the algorithm
\usepackage{algpseudocode}

% Manually remove the 'end' for some sections
\algtext*{EndIf}
\algtext*{EndFor}

% Define Left Justified Comments
\algnewcommand{\LeftComment}[1]{\Statex \(\triangleright\) #1}

% Remove the Numbering of the Algorithm
\usepackage{caption}
\DeclareCaptionLabelFormat{algnonumber}{Algorithm}
\captionsetup[algorithm]{labelformat = algnonumber}

% Define the command for a boldface instructions
\newcommand{\Is}{\textbf{ is }}
\newcommand{\To}{\textbf{ to }}
\newcommand{\Downto}{\textbf{ downto }}
\newcommand{\Or}{\textbf{ or }}
\newcommand{\And}{\textbf{ and }}
% Use them inside Math-Mode (Hence the space!)

\begin{document}
In the pseudo code that follows , we assume
that C is a set of n characters and that each character c ∈ C is an object
with an attribute c.freq giving its frequency. The algorithm builds the tree T
corresponding to the optimal code in a bottom-up manner. It begins with a set
of C leaves and performs a sequence of C − 1 merging operations to create
the final tree. The algorithm uses a min-priority queue Q, keyed on the freq
attribute, to identify the two least-frequent objects to merge together. When
we merge two objects, the result is a new object whose frequency is the sum of
the frequencies of the two objects that were merged.

\begin{algorithm}

  \caption{Huffman Coding}
  
  \begin{algorithmic}[1]
    \Statex
    
    
        \State $n: = |C|$
       \State $Q: = C$
        \For {$j \gets 1 \To n - 1$}
        
            \State$ allocate\hspace{.5em} a \hspace{.5em}new\hspace{.5em} node \hspace{.5em}z$
                \State $z.left:=x:=Extract-Min(Q)$
                \State $z.right:=:=Extract-Min(Q)$
                \State$z.freq:=x.freq +y.freq$
                \State$Insert(Q,Z)$
           
        \EndFor
        \Statex
        
        \State \Return $Extract-Min(Q)$
   
    \Statex
   
  \end{algorithmic}
  
\end{algorithm}

\end{document}
